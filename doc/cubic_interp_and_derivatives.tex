\documentclass[11pt]{article}

%%%%%%%%%%%%%%%%%%%%%%%%%%%%%%%%%%%%%%%%
% Packages
%%%%%%%%%%%%%%%%%%%%%%%%%%%%%%%%%%%%%%%%

\usepackage{oke-header-math}

%%%%%%%%%%%%%%%%%%%%%%%%%%%%%%%%%%%%%%%%
% Mathematics
%%%%%%%%%%%%%%%%%%%%%%%%%%%%%%%%%%%%%%%%

\newcommand{\sbf}{\boldsymbol{s}}
\newcommand{\pbf}{\boldsymbol{p}}
\newcommand{\qbf}{\boldsymbol{q}}

%%%%%%%%%%%%%%%%%%%%%%%%%%%%%%%%%%%%%%%%
% Title
%%%%%%%%%%%%%%%%%%%%%%%%%%%%%%%%%%%%%%%%

\title{Cubic Interpolation \& Derivatives in $d$ Dimensions}
\author{Oliver K. Ernst \\ Physics Department, UC San Diego}

%%%%%%%%%%%%%%%%%%%%%%%%%%%%%%%%%%%%%%%%
% Begin document
%%%%%%%%%%%%%%%%%%%%%%%%%%%%%%%%%%%%%%%%

\begin{document}

\maketitle

\tableofcontents

%%%%%%%%%%%%%%%%%%%%%%%%%%%%%%%%%%%%%%%%%%%%%%%%%%%%%%%%%%%%%%%%%%%%%%%%%%%%%%%%
%%%%%%%%%%%%%%%%%%%%%%%%%%%%%%%%%%%%%%%%%%%%%%%%%%%%%%%%%%%%%%%%%%%%%%%%%%%%%%%%

\section{In one dimension}

%%%%%%%%%%%%%%%%%%%%%%%%%%%%%%%%%%%%%%%%%%%%%%%%%%%%%%%%%%%%%%%%%%%%%%%%%%%%%%%%
%%%%%%%%%%%%%%%%%%%%%%%%%%%%%%%%%%%%%%%%%%%%%%%%%%%%%%%%%%%%%%%%%%%%%%%%%%%%%%%%

%%%%%%%%%%%%%%%%%%
%%%%%%%%%%%%%%%%%%

\subsection{Interpolation}

%%%%%%%%%%%%%%%%%%
%%%%%%%%%%%%%%%%%%

Consider a $1$D grid of points of length $n$, with abscissas $(s_0, \dots, s_{n-1})$.

Let values be defined on each point, given by $(t_0,\dots,t_{n-1})$.

Given a point $\tilde{x}$, define the index $i \in \{ 0,1, \dots, n-2 \}$ such that $\tilde{x}$ that falls between $s_i, s_{i+1}$.

Consider two scenarios:
\begin{enumerate}
\item \textbf{Case \#1:} the point $\tilde{x}$ is not near the boundary, i.e. $i \neq 0 \cap i \neq n-2$.

We can then define the local neighborhood of 4 values:
%---------------
\begin{equation*}
\begin{split}
p_0 &= t_{i-1} \\ 
p_1 &= t_i \\
p_2 &= t_{i+1} \\
p_3 &= t_{i+2}
\end{split}
\end{equation*}
%---------------
or more compactly
%---------------
\begin{equation*}
p_j = t_{i-1+j}
\end{equation*}
%---------------
for $j=0,1,2,3$.

How can we describe the condition for this case?

If the index $i$ falls on the lower boundary, it is $\delta_{i,0}$. A further necessary condition is that the point being queried is at that boundary, i.e. if $i=0$, then only $j=0$ is approximated; if $i=n-2$, then only $j=3$ is approximated. The condition then is $\delta_{i,0} \delta_{j,0}$.

At the other boundary, the condition is $\delta_{i,n-2} \delta_{j,3}$.

If the first condition is not met, it is $1-\delta_{i,0} \delta_{j,0}$ and $1-\delta_{i,n-2} \delta_{j,3}$.

If both are not met, it is $(1-\delta_{i,0} \delta_{j,0}) (1-\delta_{i,n-2} \delta_{j,3})$.

\item \textbf{Case \#2:} the point is near the boundary - but this only affects $p_0$ or $p_3$. The condition for this case is then: $( i = 0 \cap j = 0) \cup ( i = n-2 \cap j = 3 )$.

In this case, we must approximate the values at the boundary. The best approximation is a linear one:
%---------------
\begin{equation}
\begin{split}
p_0 &\approx 2 p_1 - p_2 = 2 t_i - t_{i+1} \\
\text{or} \qquad p_3 &\approx 2 p_2 - p_1 = 2 t_{i+1} - t_i
\end{split}
\end{equation}
%---------------

\end{enumerate}

The general formula for $p$ is then:
%---------------
\begin{equation}
\begin{split}
p_j 
= 
(1 - \delta_{i,0} \delta_{j,0} ) (1-\delta_{i,n-2} \delta_{j,3} ) t_{i-1+j}
+
\delta_{i,0} \delta_{j,0} ( 2 t_i - t_{i+1} )
+ \delta_{i,n-2} \delta_{j,3} ( 2 t_{i+1} - t_i )
\end{split}
\end{equation}
%---------------
for $j=0,1,2,3$.

Finally, let $x$ be the \textbf{fraction} that $\tilde{x}$ is between the two neighboring points, i.e. $x = (\tilde{x} - s_i) / (s_{i+1} - s_i)$.

The cubic interpolation formula is:
%---------------
\begin{equation}
f(x ; p_0, p_1, p_2, p_3) = \left ( - \frac{1}{2} p_0 + \frac{3}{2} p_1 - \frac{3}{2} p_2 + \frac{1}{2} p_3 \right ) x^3 + \left ( p_0 - \frac{5}{2} p_1 + 2 p_2 - \frac{1}{2} p_3 \right ) x^2 + \left ( - \frac{1}{2} p_0 + \frac{1}{2} p_2 \right ) x + p_1
\end{equation}
%---------------


%%%%%%%%%%%%%%%%%%
%%%%%%%%%%%%%%%%%%

\subsection{The derivative with respect to a point value}

%%%%%%%%%%%%%%%%%%
%%%%%%%%%%%%%%%%%%

Consider the derivative
%---------------
\begin{equation}
\frac{d f}{d p_k}
\end{equation}
%---------------
Assume that $p_k$ corresponds to a real point and is not approximated by other values, i.e.
%---------------
\begin{equation}
p_k = t_{i-1+k}
\end{equation}
%---------------
then:
%---------------
\begin{equation}
\frac{d f}{d p_k}
=
\sum_{j = 0}^3 \frac{\partial f}{\partial p_j} \frac{\partial p_j}{\partial p_k}
\end{equation}
%---------------
To evaluate the second term:
%---------------
\begin{equation}
\begin{split}
\frac{\partial p_j}{\partial p_k} 
=& 
\frac{\partial p_j}{\partial t_{i-1+k}} \\
=& 
(1 - \delta_{i,0} \delta_{j,0} ) (1-\delta_{i,n-2} \delta_{j,3} ) \frac{\partial t_{i-1+j}}{\partial t_{i-1+k}}
+
\delta_{i,0} \delta_{j,0} ( 2 \frac{\partial t_i}{\partial t_{i-1+k}} - \frac{\partial t_{i+1}}{\partial t_{i-1+k}} )
+ \delta_{i,n-2} \delta_{j,3} ( 2 \frac{\partial t_{i+1}}{\partial t_{i-1+k}} - \frac{\partial t_i}{\partial t_{i-1+k}} ) \\
=& 
(1 - \delta_{i,0} \delta_{j,0} ) (1-\delta_{i,n-2} \delta_{j,3} ) \delta_{k,j}
+
\delta_{i,0} \delta_{j,0} ( 2 \delta_{k,1} - \delta_{k,0} )
+ \delta_{i,n-2} \delta_{j,3} ( 2 \delta_{k,2} - \delta_{k,1} )
\end{split}
\end{equation}
%---------------
Then
%---------------
\begin{equation}
\frac{d f}{d p_k}
=
(1 - \delta_{i,0} \delta_{k,0} ) (1-\delta_{i,n-2} \delta_{k,3} ) \frac{\partial f}{\partial p_k}
+
\delta_{i,0} (2 \delta_{k,1} - \delta_{k,0} ) \frac{\partial f}{\partial p_0}
+
\delta_{i,n-2} (2 \delta_{k,2} - \delta_{k,1} ) \frac{\partial f}{\partial p_3}
\end{equation}
%---------------
The first term is the standard derivative. The delta terms in front of it ensures: (1) it exists when we are not at the boundary, (2) it vanishes when we are at the $i=0$ boundary and trying to take a derivative with respect to $k=0$ ($p_0$ is not a real point!), and (3) similar for $i=n-2$ \& $k=3$.

The remaining derivatives that appear are:
%---------------
\begin{equation}
\begin{split}
\frac{\partial f}{\partial p_0} &= - \frac{1}{2} x^3 + x^2 - \frac{1}{2} x \\
\frac{\partial f}{\partial p_1} &= \frac{3}{2} x^3 - \frac{5}{2} x^2 + 1 \\
\frac{\partial f}{\partial p_2} &= - \frac{3}{2} x^3 + 2 x^2 + \frac{1}{2} x \\
\frac{\partial f}{\partial p_3} &= \frac{1}{2} x^3 - \frac{1}{2} x^2
\end{split}
\label{eq:derivValue}
\end{equation}
%---------------

%%%%%%%%%%%%%%%%%%
%%%%%%%%%%%%%%%%%%

\subsection{The derivative with respect to $x$}

%%%%%%%%%%%%%%%%%%
%%%%%%%%%%%%%%%%%%

The derivative with respect to the point $\tilde{x}$ is:
%---------------
\begin{equation}
\begin{split}
\frac{\partial f}{\partial x} &=
3 \left ( - \frac{1}{2} p_0 + \frac{3}{2} p_1 - \frac{3}{2} p_2 + \frac{1}{2} p_3 \right ) x^2 + 2 \left ( p_0 - \frac{5}{2} p_1 + 2 p_2 - \frac{1}{2} p_3 \right ) x  - \frac{1}{2} p_0 + \frac{1}{2} p_2 \\
\frac{\partial f}{\partial \tilde{x}} 
&= \frac{\partial f}{\partial x} \frac{\partial x}{\partial \tilde{x}} = \frac{\partial f}{\partial x} \left ( s_{i+1} - s_i \right )^{-1}
\end{split}
\label{eq:derivAbs}
\end{equation}
%---------------
If the point falls near the boundary, the result is unchanged as if we make the substitutions $p_0 \approx 2 p_1 - p_2$ or $p_3 \approx 2 p_2 - p_1$ in~(\ref{eq:derivAbs}).


%%%%%%%%%%%%%%%%%%%%%%%%%%%%%%%%%%%%%%%%%%%%%%%%%%%%%%%%%%%%%%%%%%%%%%%%%%%%%%%%
%%%%%%%%%%%%%%%%%%%%%%%%%%%%%%%%%%%%%%%%%%%%%%%%%%%%%%%%%%%%%%%%%%%%%%%%%%%%%%%%

\section{In $d$-dimensions}

%%%%%%%%%%%%%%%%%%%%%%%%%%%%%%%%%%%%%%%%%%%%%%%%%%%%%%%%%%%%%%%%%%%%%%%%%%%%%%%%
%%%%%%%%%%%%%%%%%%%%%%%%%%%%%%%%%%%%%%%%%%%%%%%%%%%%%%%%%%%%%%%%%%%%%%%%%%%%%%%%

%%%%%%%%%%%%%%%%%%
%%%%%%%%%%%%%%%%%%

\subsection{Interpolation}

%%%%%%%%%%%%%%%%%%
%%%%%%%%%%%%%%%%%%

In $d$ dimensions, let the grid abscissas in each dimension $\delta = 0, \dots, d-1$ be $(a_0^{\langle \delta \rangle}, \dots, a_{n_d-1}^{\langle \delta \rangle})$ of length $n^{\langle \delta \rangle}$.

The vector of length $d$ describing the location of a grid point is then:
%---------------
\begin{equation}
\sbf^{[d] : \langle j_0,\dots,j_{d-1} \rangle} = \left ( a_{j_0}^{\langle 0 \rangle}, \dots, a_{j_{d-1}}^{\langle d-1 \rangle}  \right )
\end{equation}
%---------------
Here, we use the notation $\sbf^{[d] : \langle j_0,\dots,j_{d-1} \rangle}$ that there is one of such vector for every set of $[d]$ indexes $\langle j_{0},\dots,j_{d-1} \rangle$.

Let the values associated with each grid point be:
%---------------
\begin{equation}
t^{[d] : \langle j_0,\dots,j_{d-1} \rangle}
\end{equation}
%---------------
To further reduce the burden of indexing, define:
%---------------
\begin{equation}
[d]: \langle 0:d-1 \rangle  = [d]: \langle j_0, \dots, j_{d-1} \rangle
\end{equation}
%---------------
when specific indexes are not needed, then the abscissas and associated points are:
%---------------
\begin{equation*}
\begin{split}
\text{abscissa:}& \qquad \sbf^{[d]: \langle 0:d-1 \rangle} \\
\text{ordinate:}& \qquad t^{[d]: \langle 0:d-1 \rangle}
\end{split}
\end{equation*}
%---------------

Given a point $\tilde{\xb} = (\tilde{x}_0, \dots, \tilde{x}_{d-1})$, define the indexes $i_\delta \in \{ 0, 1, \dots, n^{\langle \delta \rangle}-2 \}$ for $\delta=0,\dots,d-1$ such that $\tilde{x}_\delta$ falls between $a_{i_\delta}^{\langle \delta \rangle}, a_{i_{\delta+1}}^{\langle \delta \rangle}$.

We want to define the values $p$ as in the 1D case. We must consider two scenarios as before:

\begin{enumerate}
%%%%%%%%%%%%%%%%%%%%%%%%%%%%%%%%%%%%%%
% CASE 1
%%%%%%%%%%%%%%%%%%%%%%%%%%%%%%%%%%%%%%
\item \textbf{Case \#1}: all points are in the interior, i.e. $i_\delta \neq 0 \cap i_\delta \neq n^{\langle \delta \rangle}-2$ for all $\delta$.

Define the values $p^{[d]: \langle 0:d-1 \rangle}$:
%---------------
\begin{equation*}
\begin{split}
p^{[d]:\langle 0, \dots, 0,0 \rangle} &= t^{[d]:\langle i_0-1, \dots, i_{d-2}-1, i_{d-1}-1 \rangle} \\
p^{[d]:\langle 0, \dots, 0,1 \rangle} &= t^{[d]:\langle i_0-1, \dots, i_{d-2}-1, i_{d-1} \rangle} \\
p^{[d]:\langle 0, \dots, 0,2 \rangle} &= t^{[d]:\langle i_0-1, \dots, i_{d-2}-1, i_{d-1}+1 \rangle} \\
p^{[d]:\langle 0, \dots, 0,3 \rangle} &= t^{[d]:\langle i_0-1, \dots, i_{d-2}-1, i_{d-1}+2 \rangle} \\
p^{[d]:\langle 0, \dots, 1,0 \rangle} &= t^{[d]:\langle i_0-1, \dots, i_{d-2}, i_{d-1}-1 \rangle} \\
p^{[d]:\langle 0, \dots, 2,0 \rangle} &= t^{[d]:\langle i_0-1, \dots, i_{d-2}+1, i_{d-1}-1 \rangle} \\
p^{[d]:\langle 0, \dots, 3,0 \rangle} &= t^{[d]:\langle i_0-1, \dots, i_{d-2}+2, i_{d-1}-1 \rangle}
\end{split}
\end{equation*}
%---------------
etc., or more generally
%---------------
\begin{equation*}
p^{[d]:\langle j_0, \dots, j_{d-1} \rangle} = t^{[d]:\langle i_0-1+j_0, \dots, i_{d-1}-1+j_{d-1} \rangle} 
\end{equation*}
%---------------
where $j=0,1,2,3$. There are $4^d$ of such points $p^{[d]: \langle 0:d-1 \rangle}$ in total.

What is the condition for this event to occur?

If all points fall near the lower boundary, it is $\prod_{\alpha = 0}^{d-1} \delta_{i_\delta,0} \delta_{j_\delta,0}$.

If no points fall near the lower boundary, it is $\prod_{\alpha = 0}^{d-1} (1 - \delta_{i_\delta,0} \delta_{j_\delta,0} )$.

Combined with the upper boundary, the full condition for no points to fall near the boundary is:
%---------------
\begin{equation*}
\prod_{\alpha = 0}^{d-1} (1 - \delta_{i_\delta,0} \delta_{j_\delta,0} ) (1 - \delta_{i_\delta,n^{\langle \delta \rangle}-2} \delta_{j_\delta,3})
\end{equation*}
%---------------

%%%%%%%%%%%%%%%%%%%%%%%%%%%%%%%%%%%%%%
% CASE 2
%%%%%%%%%%%%%%%%%%%%%%%%%%%%%%%%%%%%%%
\item \textbf{Case \#2}: at least one point falls near the boundary i.e. $( i_\delta = 0 \cap j_\delta = 0 ) \cup ( i_\delta = n^{\langle \delta \rangle}-2 \cap j_\delta = 3)$ for at least one $\delta$.

When this event occurs, some points $p$ must be approximated by others. Define the mappings acting on indexes $i_\delta$:
%---------------
\begin{equation}
\begin{split}
{\cal M} : \begin{cases}
	i_\delta \rightarrow i_\delta + 1 \quad &\text{if } i_\delta=-1 \\
	i_\delta \rightarrow i_\delta - 1 \quad &\text{if } i_\delta = n^{\langle \delta \rangle} \\
	i_\delta \rightarrow i_\delta \quad &\text{otherwise}
\end{cases} \\
{\cal P} : \begin{cases}
	i_\delta \rightarrow i_\delta + 2 \quad &\text{if } i_\delta=-1 \\
	i_\delta \rightarrow i_\delta - 2 \quad &\text{if } i_\delta = n^{\langle \delta \rangle} \\
	i_\delta \rightarrow i_\delta \quad &\text{otherwise}
\end{cases}
\end{split}
\end{equation}
%---------------
that is, ${\cal M}$ takes one step back into the lattice for any indexes that are outside, and ${\cal P}$ takes two steps back in.

We let the notation $[d]: {\cal M} \langle j_0, \dots, j_{d-1} \rangle$ denote that ${\cal M}$ is applied to all indexes, and similarly for ${\cal P}$.

If a point $t^{[d]:\langle i_0-1+j_0, \dots, i_{d-1}-1+j_{d-1} \rangle}$ is outside the lattice, then generalizing the linear approximation from the 1D case, it can be approximated as:
%---------------
\begin{equation*}
t^{[d]:\langle i_0-1+j_0, \dots, i_{d-1}-1+j_{d-1} \rangle} \approx 2 \times t^{[d]:{\cal M}\langle i_0-1+j_0, \dots, i_{d-1}-1+j_{d-1} \rangle} - t^{[d]:{\cal P}\langle i_0-1+j_0, \dots, i_{d-1}-1+j_{d-1} \rangle}
\end{equation*}
%---------------
where the mappings applied to the indexes have now referenced a valid point.

\end{enumerate}

The general formula then for the points $p$ is:
%---------------
\begin{equation}
\begin{split}
p^{[d]:\langle j_0, \dots, j_{d-1} \rangle} 
=& 
\left ( \prod_{\alpha = 0}^{d-1} (1 - \delta_{i_\delta,0} \delta_{j_\delta,0} ) (1 - \delta_{i_\delta,n^{\langle \delta \rangle}-2} \delta_{j_\delta,3} ) \right )
t^{[d]:\langle i_0-1+j_0, \dots, i_{d-1}-1+j_{d-1} \rangle} \\
&+
\left ( 1 - \prod_{\alpha = 0}^{d-1} (1 - \delta_{i_\delta,0} \delta_{j_\delta,0} ) (1 - \delta_{i_\delta,n^{\langle \delta \rangle}-2} \delta_{j_\delta,3} ) \right ) \\
&\hspace{5mm} \times \left (
2 \times t^{[d]:{\cal M}\langle i_0-1+j_0, \dots, i_{d-1}-1+j_{d-1} \rangle} - t^{[d]:{\cal P}\langle i_0-1+j_0, \dots, i_{d-1}-1+j_{d-1} \rangle}
\right )
\end{split}
\label{eq:general}
\end{equation}
%---------------
where $j=0,1,2,3$.

Finally, let $\xb$ be the fraction with components:
%---------------
\begin{equation}
x_\delta = (\tilde{x}_\delta - a_{i_\delta}^{\langle \delta \rangle}) / (a_{i_\delta+1}^{\langle \delta \rangle} - a_{i_\delta}^{\langle \delta \rangle})
\end{equation}
%---------------
for $\delta=0,\dots,d-1$.

The cubic interpolation now proceeds iteratively - define points $p^{[d-1]: \langle 0: d-2 \rangle}$:
%---------------
\begin{equation}
p^{[d-1]: \langle j_0,\dots,j_{d-2} \rangle} = f \left ( 
x_{d-1} ; 
p^{[d]: \langle j_0,\dots,j_{d-2},0 \rangle},
p^{[d]: \langle j_0,\dots,j_{d-2},1 \rangle},
p^{[d]: \langle j_0,\dots,j_{d-2},2 \rangle},
p^{[d]: \langle j_0,\dots,j_{d-2},3 \rangle}
\right )
\end{equation}
%---------------
(notice the indexes appearing on the left), for all $j=0,1,2,3$, or equivalently and more compactly:
%---------------
\begin{equation}
p^{[d-1]: \langle 0:d-2 \rangle} = f \left ( 
x_{d-1} ; 
p^{[d]: \langle 0:d-2,0 \rangle},
p^{[d]: \langle 0:d-2,1 \rangle},
p^{[d]: \langle 0:d-2,2 \rangle},
p^{[d]: \langle 0:d-2,3 \rangle}
\right )
\label{eq:iterate1}
\end{equation}
%---------------
There are $4^{d-1}$ of such points $p^{[d-1]:\langle 0:d-2 \rangle}$.

In general, the recursion relation to go from dimension $\delta+1$ to $\delta$ is:
%---------------
\begin{equation}
p^{[\delta]: \langle 0:\delta-1 \rangle} = f \left ( 
x_{\delta} ; 
p^{[\delta+1]: \langle 0:\delta-1,0 \rangle},
p^{[\delta+1]: \langle 0:\delta-1,1 \rangle},
p^{[\delta+1]: \langle 0:\delta-1,2 \rangle},
p^{[\delta+1]: \langle 0:\delta-1,3 \rangle}
\right ) 
\label{eq:rec}
\end{equation}
%---------------
for $\delta = 0,\dots,d-1$. The last iteration with $\delta = 0$ gives:
%---------------
\begin{equation}
p^{[0]} = f \left ( 
x_0 ; 
p^{[1]: \langle 0 \rangle},
p^{[1]: \langle 1 \rangle},
p^{[1]: \langle 2 \rangle},
p^{[1]: \langle 3 \rangle}
\right )
\label{eq:soln}
\end{equation}
%---------------
is the desired interpolated value we seek.

%%%%%%%%%%%%%%%%%%

\subsubsection{Pseudocode}

%%%%%%%%%%%%%%%%%%

\begin{enumerate}
\item function \textbf{iterate}($\delta$, $d$, $\xb$, $(j_0, \dots, j_{d-1})$, $p^{[d]:\langle 0:d-1 \rangle}$): \\
// This calculates $p^{[\delta]: \langle 0:\delta-1 \rangle} = $ left side of~(\ref{eq:rec})
\begin{enumerate}
\item if $\delta$ == $d$: // Arrived at $p^{[d]}$ - this we can just return
\begin{enumerate}
\item return $p^{[d]: \langle j_0, \dots, j_{d-1} \rangle}$
\end{enumerate}
\item else: // Calculate the right side of~(\ref{eq:rec})
\begin{enumerate}
\item $p^{[\delta+1]: \langle 0: \delta-1,0 \rangle} = $ \textbf{iterate}($\delta+1$, $d$, $\xb$, $( j_0, \dots, j_{\delta}=0, \dots, j_{d-1} )$, $p^{[d]:\langle 0:d-1 \rangle}$)
\item $p^{[\delta+1]: \langle 0: \delta-1,1 \rangle} = $ \textbf{iterate}($\delta+1$, $d$, $\xb$, $( j_0, \dots, j_{\delta}=1, \dots, j_{d-1} )$, $p^{[d]:\langle 0:d-1 \rangle}$)
\item $p^{[\delta+1]: \langle 0: \delta-1,2 \rangle} = $ \textbf{iterate}($\delta+1$, $d$, $\xb$, $( j_0, \dots, j_{\delta}=2, \dots, j_{d-1} )$, $p^{[d]:\langle 0:d-1 \rangle}$)
\item $p^{[\delta+1]: \langle 0: \delta-1,3 \rangle} = $ \textbf{iterate}($\delta+1$, $d$, $\xb$, $( j_0, \dots, j_{\delta}=3, \dots, j_{d-1} )$, $p^{[d]:\langle 0:d-1 \rangle}$)
\item return $f \left ( 
x_{\delta} ; 
p^{[\delta+1]: \langle 0:\delta-1,0 \rangle},
p^{[\delta+1]: \langle 0:\delta-1,1 \rangle},
p^{[\delta+1]: \langle 0:\delta-1,2 \rangle},
p^{[\delta+1]: \langle 0:\delta-1,3 \rangle}
\right ) $ // $= p^{[\delta]}$
\end{enumerate}
\end{enumerate}

\item To start: \textbf{iterate}(0,$d$,$\xb$,$(j_0,\dots,j_{d-1})$,$p^{[d]:\langle 0:d-1 \rangle}$)
\end{enumerate}

%%%%%%%%%%%%%%%%%%
%%%%%%%%%%%%%%%%%%

\subsection{The derivative with respect to a point value}

%%%%%%%%%%%%%%%%%%
%%%%%%%%%%%%%%%%%%

What is the derivative with respect to a point value? 

Let the point to differentiate with respect to be:
%---------------
\begin{equation}
p^{[d]: \langle k_0, \dots, k_{d-1} \rangle}
\end{equation}
%---------------
where $k=0,1,2,3$. We assume that by definition, we are differentiating with respect to a real lattice point, not one which is outside the lattice. In terms of the indexes $i_\delta$, this can then be written as:
%---------------
\begin{equation}
p^{[d]: \langle k_0, \dots, k_{d-1} \rangle} = t^{[d]: \langle i_0-1+k_0, \dots, i_{d-1}-1+k_{d-1} \rangle}
\end{equation}
%---------------
Then we seek:
%---------------
\begin{equation}
\frac{\partial p^{[0]}}{\partial p^{[d]: \langle k_0, \dots, k_{d-1} \rangle}}
\end{equation}
%---------------

Differentiating the recursion~(\ref{eq:rec}) and using the chain rule gives:
%---------------
\begin{equation}
\frac{\partial p^{[\delta]: \langle 0:\delta-1 \rangle} }{\partial p^{[d]: \langle k_0, \dots, k_{d-1} \rangle}} = \sum_{j_\delta=0}^3 
\frac{ \partial f \left ( 
x_{\delta} ; 
p^{[\delta+1]: \langle 0:\delta-1,0 \rangle},
p^{[\delta+1]: \langle 0:\delta-1,1 \rangle},
p^{[\delta+1]: \langle 0:\delta-1,2 \rangle},
p^{[\delta+1]: \langle 0:\delta-1,3 \rangle}
\right ) 
}{
\partial p^{[\delta+1]: \langle 0:\delta-1,j_\delta \rangle}
}
\frac{\partial p^{[\delta+1]: \langle 0:\delta-1,j_\delta \rangle}
}{
\partial p^{[d]: \langle k_0, \dots, k_{d-1} \rangle}
}
\end{equation}
%---------------
for $\delta = 0,\dots,d-2$.

The first term can be evaluated using~(\ref{eq:derivValue}). We immediately notice an important property: the interpolation is linear in the point values, such that the first term does not depend on them. This greatly reduces the complexity - we use the notation from~(\ref{eq:derivValue}):
%---------------
\begin{equation}
\frac{ 
\partial f(x_\delta)
}{
\partial p_{j_\delta}
}
\end{equation}
%---------------
to denote the derivative, giving:
%---------------
\begin{equation}
\frac{\partial p^{[\delta]: \langle 0:\delta-1 \rangle} }{\partial p^{[d]: \langle k_0, \dots, k_{d-1} \rangle}} 
= 
\sum_{j_\delta=0}^3 
\frac{
\partial f(x_\delta)
}{
\partial p_{j_\delta}
}
\frac{\partial p^{[\delta+1]: \langle 0:\delta-1,j_\delta \rangle}
}{
\partial p^{[d]: \langle k_0, \dots, k_{d-1} \rangle}
}
\label{eq:recDerivP}
\end{equation}
%---------------

With $\delta=0$ and using the recursion $d-1$ times gives:
%---------------
\begin{equation}
\frac{\partial p^{[0]}}{\partial p^{[d]: \langle k_0, \dots, k_{d-1} \rangle}}
=
\sum_{j_0,\dots,j_{d-1}} 
\left (
\prod_{\alpha=0}^{d-1} 
\frac{
\partial f(x_\alpha)
}{
\partial p_{j_\alpha}
}
\right ) 
\left (
\frac{
\partial p^{[d]: \langle j_0, \dots, j_{d-1} \rangle}
}{
\partial p^{[d]: \langle k_0, \dots, k_{d-1} \rangle}
}
\right )
\end{equation}
%---------------

It is tempting to simply stick in delta functions for the term on the right; however we must be careful and use~(\ref{eq:general}) instead:
%---------------
\begin{equation}
\begin{split}
%%%%%%%%%%%%%%%%%%%%%%%%%%%%%%%%
% line 1
\frac{
\partial p^{[d]: \langle j_0, \dots, j_{d-1} \rangle}
}{
\partial p^{[d]: \langle k_0, \dots, k_{d-1} \rangle}
}
=&
\frac{
\partial p^{[d]: \langle j_0, \dots, j_{d-1} \rangle}
}{
\partial t^{[d]: \langle i_0-1+k_0, \dots, i_{d-1}-1+k_{d-1} \rangle}
} \\
%%%%%%%%%%%%%%%%%%%%%%%%%%%%%%%%
% line 2
=&
\left ( \prod_{\alpha = 0}^{d-1} (1 - \delta_{i_\delta,0} \delta_{j_\delta,0} ) (1 - \delta_{i_\delta,n^{\langle \delta \rangle}-2} \delta_{j_\delta,3} ) \right )
\frac{
\partial t^{[d]:\langle i_0-1+j_0, \dots, i_{d-1}-1+j_{d-1} \rangle}
}{
\partial t^{[d]: \langle i_0-1+k_0, \dots, i_{d-1}-1+k_{d-1} \rangle}
} \\
%%%%%%%%%%%%%%%%%%%%%%%%%%%%%%%%
% line 3
& +
\left ( 1 - \prod_{\alpha = 0}^{d-1} (1 - \delta_{i_\delta,0} \delta_{j_\delta,0} ) (1 - \delta_{i_\delta,n^{\langle \delta \rangle}-2} \delta_{j_\delta,3} ) \right ) \\
%%%%%%%%%%%%%%%%%%%%%%%%%%%%%%%%
% line 4
& \hspace{5mm} \times
\left (
2 \times 
\frac{
\partial t^{[d]:{\cal M}\langle i_0-1+j_0, \dots, i_{d-1}-1+j_{d-1} \rangle}
}{
\partial t^{[d]: \langle i_0-1+k_0, \dots, i_{d-1}-1+k_{d-1} \rangle}
}
-
\frac{
\partial t^{[d]:{\cal P}\langle i_0-1+j_0, \dots, i_{d-1}-1+j_{d-1} \rangle}
}{
\partial t^{[d]: \langle i_0-1+k_0, \dots, i_{d-1}-1+k_{d-1} \rangle}
}
\right ) \\
%%%%%%%%%%%%%%%%%%%%%%%%%%%
% line 5
=&
\left ( \prod_{\alpha = 0}^{d-1} (1 - \delta_{i_\delta,0} \delta_{j_\delta,0} ) (1 - \delta_{i_\delta,n^{\langle \delta \rangle}-2} \delta_{j_\delta,3}) \right )
\left (
\prod_{\alpha=0}^{d-1} \delta_{k_\delta, j_\delta}
\right ) \\
%%%%%%%%%%%%%%%%%%%%%%%%%%%
% line 6
& +
\left ( 1 - \prod_{\alpha = 0}^{d-1} (1 - \delta_{i_\delta,0} \delta_{j_\delta,0} ) (1 - \delta_{i_\delta,n^{\langle \delta \rangle}-2} \delta_{j_\delta,3}) \right ) \\
%%%%%%%%%%%%%%%%%%%%%%%%%%%
% line 7
& \hspace{5mm} \times
\left (
2 
\prod_{\alpha=0}^{d-1} \delta_{i_\delta-1+k_\delta, {\cal M}(i_\delta-1+j_\delta) }
-
\prod_{\alpha=0}^{d-1} \delta_{i_\delta-1+k_\delta, {\cal P}(i_\delta-1+j_\delta) }
\right ) 
\end{split}
\label{eq:shit}
\end{equation}
%---------------

The first term is easy, as it just picks out $j_\delta = k_\delta$ in the sums. To proceed in simplifying the second term, we would need the inverse mappings ${\cal M}^{-1}$, ${\cal P}^{-1}$ - but these don't exist. 

We are left with:
%---------------
\begin{equation}
\begin{split}	
% line 1
\frac{\partial p^{[0]}}{\partial p^{[d]: \langle k_0, \dots, k_{d-1} \rangle}}
=&
\left ( \prod_{\alpha = 0}^{d-1} (1 - \delta_{i_\delta,0}\delta_{j_\delta,0}) (1 - \delta_{i_\delta,n^{\langle \delta \rangle}-2}\delta_{j_\delta,3}) \right )
\left (
\prod_{\alpha=0}^{d-1} 
\frac{
\partial f(x_\alpha)
}{
\partial p_{k_\alpha}
}
\right ) \\
% line 2
&+ \left ( 1 - \prod_{\alpha = 0}^{d-1} (1 - \delta_{i_\delta,0}\delta_{j_\delta,0}) (1 - \delta_{i_\delta,n^{\langle \delta \rangle}-2}\delta_{j_\delta,3}) \right ) \\
% line 3
&\hspace{5mm} \times
\sum_{j_0, \dots, j_{d-1}}
% prod 1
\left (
\prod_{\alpha=0}^{d-1} 
\frac{
\partial f(x_\alpha)
}{
\partial p_{j_\alpha}
}
\right ) 
% prod 2
\left (
2 
\prod_{\alpha=0}^{d-1} \delta_{i_\delta-1+k_\delta, {\cal M}(i_\delta-1+j_\delta) }
-
\prod_{\alpha=0}^{d-1} \delta_{i_\delta-1+k_\delta, {\cal P}(i_\delta-1+j_\delta) }
\right ) 
\end{split}
\end{equation}
%---------------

The algorithm here is therefore split:
\begin{enumerate}
% Case 1
\item Case \#1: the point is interior, i.e. away from the boundary in \textbf{all} dimensions, i.e. $i_\delta \neq 0 \cap i_\delta \neq n^{\langle \delta \rangle} -2$ for \textbf{all} $\delta=0,\dots,d-1$.

Here the case easy, and does not require recursion! The answer is:
%---------------
\begin{equation}
\prod_{\alpha=0}^{d-1} 
\frac{
\partial f(x_\alpha)
}{
\partial p_{k_\alpha}
}
\end{equation}
%---------------
which can be evaluated using~(\ref{eq:derivValue}).

% Case 2
\item Case \# 2: here there is no apparent simplification, and we must resort to a recursive approach to evaluate~(\ref{eq:recDerivP}).

\end{enumerate}

%%%%%%%%%%%%%%%%%%

\subsubsection{Pseudocode for case \#2}

%%%%%%%%%%%%%%%%%%

\begin{enumerate}
\item function \textbf{iterate\_deriv\_p}($\delta$,$d$,$(j_0, \dots, j_{d-1})$,$p^{[d]:\langle 0:d-1 \rangle}$): \\
// This evaluates $\partial p^{[\delta]:\langle 0: \delta -1 \rangle} / \partial p^{[d] : \langle k_0, \dots, k_{d-1} \rangle} = $ left side of~(\ref{eq:recDerivP})
\begin{enumerate}
\item if $\delta == d$: // We have a complete set of idxs to evaluate~(\ref{eq:shit})
\begin{enumerate}
\item if $(i_\delta=0 \cap j_\delta =0) \cup (i_\delta=n^{\langle \delta \rangle} -2 \cap j_\delta = 3)$ for \textbf{at least} one $\delta=0,\dots,d-1$:
\begin{enumerate}
\item if ${\cal M} \langle i_0 - 1 + j_0 : i_{d-1} - 1 + j_{d-1} \rangle == \langle i_0 - 1 + k_0 : i_{d-1} - 1 + k_{d-1} \rangle$: return 2
\item if ${\cal P} \langle i_0 - 1 + j_0 : i_{d-1} - 1 + j_{d-1} \rangle == \langle i_0 - 1 + k_0 : i_{d-1} - 1 + k_{d-1} \rangle$: return -1
\item else: return 0
\end{enumerate}
\item else:
\begin{enumerate}
\item if $\langle j_0, \dots, j_{d-1} \rangle = \langle k_0, \dots, k_{d-1} \rangle$: return 1
\item else: return 0
\end{enumerate}
\end{enumerate}
\item else: // Go deeper
\begin{enumerate}
\item $\partial p^{[\delta+1]:\langle 0: \delta-1, 0 \rangle} / \partial p^{[d] : \langle k_0, \dots, k_{d-1} \rangle}$ = \textbf{iterate\_deriv\_p}($\delta$,$d$,$(j_0, \dots, j_\delta=0, \dots, j_{d-1})$,$p^{[d]:\langle 0:d-1 \rangle}$)
\item $\partial p^{[\delta+1]:\langle 0: \delta-1, 1 \rangle} / \partial p^{[d] : \langle k_0, \dots, k_{d-1} \rangle}$ = \textbf{iterate\_deriv\_p}($\delta$,$d$,$(j_0, \dots, j_\delta=1, \dots, j_{d-1})$,$p^{[d]:\langle 0:d-1 \rangle}$)
\item $\partial p^{[\delta+1]:\langle 0: \delta-1, 2 \rangle} / \partial p^{[d] : \langle k_0, \dots, k_{d-1} \rangle}$ = \textbf{iterate\_deriv\_p}($\delta$,$d$,$(j_0, \dots, j_\delta=2, \dots, j_{d-1})$,$p^{[d]:\langle 0:d-1 \rangle}$)
\item $\partial p^{[\delta+1]:\langle 0: \delta-1, 3 \rangle} / \partial p^{[d] : \langle k_0, \dots, k_{d-1} \rangle}$ = \textbf{iterate\_deriv\_p}($\delta$,$d$,$(j_0, \dots, j_\delta=3, \dots, j_{d-1})$,$p^{[d]:\langle 0:d-1 \rangle}$)
\item return $\sum_{j_\delta=0}^3 
\frac{
\partial f(x_\delta)
}{
\partial p_{j_\delta}
}
\frac{\partial p^{[\delta+1]: \langle 0:\delta-1,j_\delta \rangle}
}{
\partial p^{[d]: \langle k_0, \dots, k_{d-1} \rangle}
}
$
\end{enumerate}
\end{enumerate}

\item To start: \textbf{iterate\_deriv\_p}(0,$d$,$(j_0, \dots, j_{d-1})$,$p^{[d]:\langle 0:d-1 \rangle}$)
\end{enumerate}



%%%%%%%%%%%%%%%%%%
%%%%%%%%%%%%%%%%%%

\subsection{The derivative with respect to $x$}

%%%%%%%%%%%%%%%%%%
%%%%%%%%%%%%%%%%%%

We seek the derivative with respect to the $k$-th component $x_k$ of $\xb$:
%---------------
\begin{equation}
\frac{\partial p^{[0]}}{\partial x_k}
\end{equation}
%---------------
Consider generally differentiating the left side of the recursion relation~(\ref{eq:rec}):
%---------------
\begin{equation}
\frac{\partial p^{[\delta]: \langle 0:\delta-1 \rangle} }{\partial x_k} 
\end{equation}
%---------------
If $k = \delta$, then this may be evaluated using:
%---------------
\begin{equation}
\frac{\partial p^{[k]: \langle 0:\delta \rangle}
}{
\partial x_k
}
=
\frac{\partial 
f \left ( 
x_{k} ; 
p^{[k+1]: \langle 0:k-1,0 \rangle},
p^{[k+1]: \langle 0:k-1,1 \rangle},
p^{[k+1]: \langle 0:k-1,2 \rangle},
p^{[k+1]: \langle 0:k-1,3 \rangle}
\right ) 
}{
\partial x_k
}
\label{eq:evalk}
\end{equation}
%---------------
where the points $p^{[k+1]: \langle 0:k \rangle}$ on the right hand side must be evaluated using~(\ref{eq:rec}) as before.

If $k < \delta$, then we must evaluate the derivative using~(\ref{eq:rec}):
%---------------
\begin{equation}
\frac{\partial p^{[\delta]: \langle 0:\delta-1 \rangle} }{\partial x_k} 
=
\sum_{j_\delta=0}^3 
\frac{
\partial f(x_\delta)
}{
\partial p_{j_\delta}
}
\frac{\partial p^{[\delta+1]: \langle 0:\delta \rangle}
}{
\partial x_k
}
\label{eq:recDer}
\end{equation}
%---------------

Starting at $\delta=0$, we must therefore apply the recursion~(\ref{eq:recDer}) $k$ times:
%---------------
\begin{equation}
\frac{\partial p^{[0]} }{\partial x_k} 
=
\sum_{j_0, \dots, j_{k-1}}
\left (
\prod_{\alpha=0}^{k-1}
\frac{
\partial f(x_\alpha)
}{
\partial p_{j_\alpha}
}
\right )
\frac{\partial p^{[k]: \langle 0:\delta \rangle}
}{
\partial x_k
}
\end{equation}
%---------------
and then use~(\ref{eq:evalk}):
%---------------
\begin{equation}
\frac{\partial p^{[0]} }{\partial x_k} 
=
\sum_{j_0, \dots, j_{k-1}}
\left (
\prod_{\alpha=0}^{k-1}
\frac{
\partial f(x_\alpha)
}{
\partial p_{j_\alpha}
}
\right )
\frac{\partial 
f \left ( 
x_{k} ; 
p^{[k+1]: \langle 0:k-1,0 \rangle},
p^{[k+1]: \langle 0:k-1,1 \rangle},
p^{[k+1]: \langle 0:k-1,2 \rangle},
p^{[k+1]: \langle 0:k-1,3 \rangle}
\right ) 
}{
\partial x_k
}
\end{equation}
%---------------
This can be evaluated using the 1D result~(\ref{eq:derivAbs}) for the second term; unfortunately, this requires $d- 1 - k$ levels of further recursion~(\ref{eq:rec}) to determine the $p^{[k+1]: \langle 0:k \rangle}$.

Also: do not forget that:
%---------------
\begin{equation}
\frac{\partial p^{[0]}}{\partial \tilde{x}_k} 
= \frac{\partial p^{[0]}}{\partial x_k} \frac{\partial x_k}{\partial \tilde{x}_k} 
= \frac{\partial p^{[0]}}{\partial x_k} \left ( s_{i_k+1}^{\langle k \rangle} - s_{i_k}^{\langle k \rangle} \right )^{-1}
\end{equation}
%---------------
since $\xb$ refers to a fraction between $0,1$.

%%%%%%%%%%%%%%%%%%

\subsubsection{Pseudocode}

%%%%%%%%%%%%%%%%%%

\begin{enumerate}
\item function \textbf{iterate\_deriv\_x}($\delta$,$k$,$d$,$\xb$,$(j_0, \dots, j_{k-1})$, $p^{[d]:\langle 0:d-1 \rangle}$): \\
// This evaluates $\partial p^{[\delta]: \langle 0:\delta-1 \rangle} / \partial x_k = $ the left hand of~(\ref{eq:recDer})
\begin{enumerate}
\item if $\delta == k$: // Evaluate using~(\ref{eq:evalk})
\begin{enumerate}
\item $p^{[k+1]: \langle 0:k-1,0 \rangle}$ = \textbf{iterate}($k+1$, $d$, $\xb$, $(j_{0}, \dots, j_{k-1}, j_k=0)$, $p^{[d]:\langle 0:d-1 \rangle}$)
\item $p^{[k+1]: \langle 0:k-1,1 \rangle}$ = \textbf{iterate}($k+1$, $d$, $\xb$, $(j_{0}, \dots, j_{k-1}, j_k=1)$, $p^{[d]:\langle 0:d-1 \rangle}$)
\item $p^{[k+1]: \langle 0:k-1,2 \rangle}$ = \textbf{iterate}($k+1$, $d$, $\xb$, $(j_{0}, \dots, j_{k-1}, j_k=2)$, $p^{[d]:\langle 0:d-1 \rangle}$)
\item $p^{[k+1]: \langle 0:k-1,3 \rangle}$ = \textbf{iterate}($k+1$, $d$, $\xb$, $(j_{0}, \dots, j_{k-1}, j_k=3)$, $p^{[d]:\langle 0:d-1 \rangle}$)
\item return $\partial
f \left ( 
x_{k} ; 
p^{[k+1]: \langle 0:k-1,0 \rangle},
p^{[k+1]: \langle 0:k-1,1 \rangle},
p^{[k+1]: \langle 0:k-1,2 \rangle},
p^{[k+1]: \langle 0:k-1,3 \rangle}
\right ) 
/
\partial x_k
$ // using~(\ref{eq:derivAbs})
\end{enumerate}
\item else: // Recurse using~(\ref{eq:recDer})
\begin{enumerate}
\item $\partial p^{[\delta+1]: \langle 0:\delta-1,0 \rangle} / \partial x_k$ = \textbf{iterate\_deriv\_x}($\delta+1$,$k$,$d$,$\xb$,$(j_0, \dots, j_{\delta} = 0,\dots, j_{k-1})$,$p^{[d]:\langle 0:d-1 \rangle}$)
\item $\partial p^{[\delta+1]: \langle 0:\delta-1,1 \rangle} / \partial x_k$ = \textbf{iterate\_deriv\_x}($\delta+1$,$k$,$d$,$\xb$,$(j_0, \dots, j_{\delta} = 1,\dots, j_{k-1})$,$p^{[d]:\langle 0:d-1 \rangle}$)
\item $\partial p^{[\delta+1]: \langle 0:\delta-1,2 \rangle} / \partial x_k$ = \textbf{iterate\_deriv\_x}($\delta+1$,$k$,$d$,$\xb$,$(j_0, \dots, j_{\delta} = 2,\dots, j_{k-1})$,$p^{[d]:\langle 0:d-1 \rangle}$)
\item $\partial p^{[\delta+1]: \langle 0:\delta-1,3 \rangle} / \partial x_k$ = \textbf{iterate\_deriv\_x}($\delta+1$,$k$,$d$,$\xb$,$(j_0, \dots, j_{\delta} = 3,\dots, j_{k-1})$,$p^{[d]:\langle 0:d-1 \rangle}$)
\item return $
\sum_{j_\delta=0}^3 
\frac{
\partial f(x_\delta)
}{
\partial p_{j_\delta}
}
\frac{\partial p^{[\delta+1]: \langle 0:\delta \rangle}
}{
\partial x_k
}
$ // = right side of~(\ref{eq:recDer})
\end{enumerate}
\end{enumerate}

\item To start: \textbf{iterate\_deriv\_x}($0$,$k$,$d$,$\xb$,$(j_0, \dots, j_{k-1})$,$p^{[d]:\langle 0:d-1 \rangle}$)
\end{enumerate}

\end{document}


