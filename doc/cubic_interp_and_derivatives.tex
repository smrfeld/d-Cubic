\documentclass[11pt]{article}

%%%%%%%%%%%%%%%%%%%%%%%%%%%%%%%%%%%%%%%%
% Packages
%%%%%%%%%%%%%%%%%%%%%%%%%%%%%%%%%%%%%%%%

\usepackage{oke-header-math}

%%%%%%%%%%%%%%%%%%%%%%%%%%%%%%%%%%%%%%%%
% Mathematics
%%%%%%%%%%%%%%%%%%%%%%%%%%%%%%%%%%%%%%%%

\newcommand{\sbf}{\boldsymbol{s}}
\newcommand{\pbf}{\boldsymbol{p}}
\newcommand{\qbf}{\boldsymbol{q}}

%%%%%%%%%%%%%%%%%%%%%%%%%%%%%%%%%%%%%%%%
% Title
%%%%%%%%%%%%%%%%%%%%%%%%%%%%%%%%%%%%%%%%

\title{Cubic Interpolation \& Derivatives in $d$ Dimensions}
\author{Oliver K. Ernst}

%%%%%%%%%%%%%%%%%%%%%%%%%%%%%%%%%%%%%%%%
% Begin document
%%%%%%%%%%%%%%%%%%%%%%%%%%%%%%%%%%%%%%%%

\begin{document}

\maketitle

\tableofcontents

%%%%%%%%%%%%%%%%%%%%%%%%%%%%%%%%%%%%%%%%%%%%%%%%%%%%%%%%%%%%%%%%%%%%%%%%%%%%%%%%
%%%%%%%%%%%%%%%%%%%%%%%%%%%%%%%%%%%%%%%%%%%%%%%%%%%%%%%%%%%%%%%%%%%%%%%%%%%%%%%%

\section{In one dimension}

%%%%%%%%%%%%%%%%%%%%%%%%%%%%%%%%%%%%%%%%%%%%%%%%%%%%%%%%%%%%%%%%%%%%%%%%%%%%%%%%
%%%%%%%%%%%%%%%%%%%%%%%%%%%%%%%%%%%%%%%%%%%%%%%%%%%%%%%%%%%%%%%%%%%%%%%%%%%%%%%%

%%%%%%%%%%%%%%%%%%
%%%%%%%%%%%%%%%%%%

\subsection{Interpolation}

%%%%%%%%%%%%%%%%%%
%%%%%%%%%%%%%%%%%%

Consider a $1$D grid of points of length $n$, with abscissas $(s_0, \dots, s_{n-1})$.

Let values be defined on each point, given by $(t_0,\dots,t_{n-1})$.

Given a point $\tilde{x}$ that falls between two abscissas $s_i, s_{i+1}$, define the surrounding values:

$p_0 = t_{i-1}, p_1 = t_i, p_2 = t_{i+1}, p_3 = t_{i+2}$. 

Let $x$ be the \textbf{fraction} that $\tilde{x}$ is between the two neighboring points, i.e. $x = (\tilde{x} - s_i) / (s_{i+1} - s_i)$.

The cubic interpolation formula is:
%---------------
\begin{equation}
f(x ; p_0, p_1, p_2, p_3) = \left ( - \frac{1}{2} p_0 + \frac{3}{2} p_1 - \frac{3}{2} p_2 + \frac{1}{2} p_3 \right ) x^3 + \left ( p_0 - \frac{5}{2} p_1 + 2 p_2 - \frac{1}{2} p_3 \right ) x^2 + \left ( - \frac{1}{2} p_0 + \frac{1}{2} p_2 \right ) x + p_1
\end{equation}
%---------------

Here, assume the case where the point does not fall near the boundary, i.e. all the grid points $p_0,p_1,p_2,p_3$ exist (else they need to be approximated).

%%%%%%%%%%%%%%%%%%
%%%%%%%%%%%%%%%%%%

\subsection{The derivative with respect to a point value}

%%%%%%%%%%%%%%%%%%
%%%%%%%%%%%%%%%%%%

The derivatives with respect to a grid point values are:
%---------------
\begin{equation}
\begin{split}
\frac{\partial f}{\partial p_0} &= - \frac{1}{2} x^3 + x^2 - \frac{1}{2} x \\
\frac{\partial f}{\partial p_1} &= \frac{3}{2} x^3 - \frac{5}{2} x^2 + 1 \\
\frac{\partial f}{\partial p_2} &= - \frac{3}{2} x^3 + 2 x^2 + \frac{1}{2} x \\
\frac{\partial f}{\partial p_3} &= \frac{1}{2} x^3 - \frac{1}{2} x^2
\end{split}
\label{eq:derivValue}
\end{equation}
%---------------

%%%%%%%%%%%%%%%%%%
%%%%%%%%%%%%%%%%%%

\subsection{The derivative with respect to $x$}

%%%%%%%%%%%%%%%%%%
%%%%%%%%%%%%%%%%%%

The derivative with respect to the point $\tilde{x}$ is:
%---------------
\begin{equation}
\begin{split}
\frac{\partial f}{\partial x} &=
3 \left ( - \frac{1}{2} p_0 + \frac{3}{2} p_1 - \frac{3}{2} p_2 + \frac{1}{2} p_3 \right ) x^2 + 2 \left ( p_0 - \frac{5}{2} p_1 + 2 p_2 - \frac{1}{2} p_3 \right ) x  - \frac{1}{2} p_0 + \frac{1}{2} p_2 \\
\frac{\partial f}{\partial \tilde{x}} 
&= \frac{\partial f}{\partial x} \frac{\partial x}{\partial \tilde{x}} = \frac{\partial f}{\partial x} \left ( s_{i+1} - s_i \right )^{-1}
\end{split}
\label{eq:derivAbs}
\end{equation}
%---------------

%%%%%%%%%%%%%%%%%%%%%%%%%%%%%%%%%%%%%%%%%%%%%%%%%%%%%%%%%%%%%%%%%%%%%%%%%%%%%%%%
%%%%%%%%%%%%%%%%%%%%%%%%%%%%%%%%%%%%%%%%%%%%%%%%%%%%%%%%%%%%%%%%%%%%%%%%%%%%%%%%

\section{In $d$-dimensions}

%%%%%%%%%%%%%%%%%%%%%%%%%%%%%%%%%%%%%%%%%%%%%%%%%%%%%%%%%%%%%%%%%%%%%%%%%%%%%%%%
%%%%%%%%%%%%%%%%%%%%%%%%%%%%%%%%%%%%%%%%%%%%%%%%%%%%%%%%%%%%%%%%%%%%%%%%%%%%%%%%

%%%%%%%%%%%%%%%%%%
%%%%%%%%%%%%%%%%%%

\subsection{Interpolation}

%%%%%%%%%%%%%%%%%%
%%%%%%%%%%%%%%%%%%

In $d$ dimensions, let the grid abscissas in each dimension be $(s_0^d, \dots, s_{n_d-1}^d)$ of length $n_d$.

The grid points are then:
%---------------
\begin{equation}
\sbf^{[d] : \langle j_1,\dots,j_d \rangle} = \left ( s_{j_1}^1, \dots, s_{j_d}^d  \right )
\end{equation}
%---------------
Here, we use the notation $\sbf^{[d] : \langle j_1,\dots,j_d \rangle}$ that there is one of such vector (of dimension $d$) for every set of $d$ indexes $\langle j_1,\dots,j_d \rangle$.

Let the values associated with each grid point be:
%---------------
\begin{equation}
t^{[d] : \langle j_1,\dots,j_d \rangle}
\end{equation}
%---------------
To further reduce the burden of indexing, define:
%---------------
\begin{equation}
[d]: \langle 1:d \rangle  = [d]: \langle j_1, \dots, j_d \rangle
\end{equation}
%---------------
when specific indexes are not needed, then the abscissas and associated points are:
%---------------
\begin{equation*}
\begin{split}
\sbf^{[d]: \langle 1:d \rangle} \\
t^{[d]: \langle 1:d \rangle}
\end{split}
\end{equation*}
%---------------

Given a point $\tilde{\xb} = (\tilde{x}^1, \dots, \tilde{x}^d)$ of dimension $d$ that falls between grid points $s_{i_d}^d, s_{i_{d+1}}^d$ in each dimension, define the values $p^{[d]: \langle 1:d \rangle}$:
%---------------
\begin{equation*}
\begin{split}
p^{[d]:\langle 0, \dots, 0,0 \rangle} &= t^{[d]:\langle i_1-1, \dots, i_{d-1}-1, i_d-1 \rangle} \\
p^{[d]:\langle 0, \dots, 0,1 \rangle} &= t^{[d]:\langle i_1-1, \dots, i_{d-1}-1, i_d \rangle} \\
p^{[d]:\langle 0, \dots, 0,2 \rangle} &= t^{[d]:\langle i_1-1, \dots, i_{d-1}-1, i_d+1 \rangle} \\
p^{[d]:\langle 0, \dots, 0,3 \rangle} &= t^{[d]:\langle i_1-1, \dots, i_{d-1}-1, i_d+2 \rangle} \\
p^{[d]:\langle 0, \dots, 1,0 \rangle} &= t^{[d]:\langle i_1-1, \dots, i_{d-1}, i_d-1 \rangle} \\
p^{[d]:\langle 0, \dots, 2,0 \rangle} &= t^{[d]:\langle i_1-1, \dots, i_{d-1}+1, i_d-1 \rangle} \\
p^{[d]:\langle 0, \dots, 3,0 \rangle} &= t^{[d]:\langle i_1-1, \dots, i_{d-1}+2, i_d-1 \rangle}
\end{split}
\end{equation*}
%---------------
etc., or more generally
%---------------
\begin{equation*}
p^{[d]:\langle j_1, \dots, j_d \rangle} = t^{[d]:\langle i_1-1+j_1, \dots, i_d-1+j_d \rangle} 
\end{equation*}
%---------------
where $j=0,1,2,3$.

There are $4^d$ of such points $p^{[d]: \langle 1:d \rangle}$ in total.

Let $\xb$ be the fraction with components:
%---------------
\begin{equation}
x_\delta = (\tilde{x}^\delta - s_{i_\delta}^\delta) / (s_{i_\delta+1}^\delta - s_{i_\delta}^\delta)
\end{equation}
%---------------
for $\delta=1,\dots,d$.

The cubic interpolation now proceeds iteratively - define points $p^{[d-1]: \langle 2: d \rangle}$:
%---------------
\begin{equation}
p^{[d-1]: \langle j_2,\dots,j_d \rangle} = f \left ( 
x_1 ; 
p^{[d]: \langle 0,j_2,\dots,j_d \rangle},
p^{[d]: \langle 1,j_2,\dots,j_d \rangle},
p^{[d]: \langle 2,j_2,\dots,j_d \rangle},
p^{[d]: \langle 3,j_2,\dots,j_d \rangle}
\right )
\end{equation}
%---------------
(notice the indexes appearing on the left), for all $i=0,1,2,3$, or equivalently and more compactly:
%---------------
\begin{equation}
p^{[d-1]: \langle 2:d \rangle} = f \left ( 
x_1 ; 
p^{[d]: \langle 0,2:d \rangle},
p^{[d]: \langle 1,2:d \rangle},
p^{[d]: \langle 2,2:d \rangle},
p^{[d]: \langle 3,2:d \rangle}
\right )
\label{eq:iterate1}
\end{equation}
%---------------

There are $4^{d-1}$ of such points $p^{[d-1]:\langle 2:d \rangle}$.

In general, the recursion relation to go from dimension $\delta$ to $\delta-1$ is:
%---------------
\begin{equation}
p^{[\delta-1]: \langle d-\delta+2:d \rangle} = f \left ( 
x_{d-\delta+1} ; 
p^{[\delta]: \langle 0,d-\delta+2:d \rangle},
p^{[\delta]: \langle 1,d-\delta+2:d \rangle},
p^{[\delta]: \langle 2,d-\delta+2:d \rangle},
p^{[\delta]: \langle 3,d-\delta+2:d \rangle}
\right )
\label{eq:rec}
\end{equation}
%---------------

We can continue this way until we reach the the last dimension $d=0$, and are out of indexes on the left:
%---------------
\begin{equation}
p^{[0]} = f \left ( 
x_d ; 
p^{[1]: \langle 0 \rangle},
p^{[1]: \langle 1 \rangle},
p^{[1]: \langle 2 \rangle},
p^{[1]: \langle 3 \rangle}
\right )
\label{eq:soln}
\end{equation}
%---------------
is the desired interpolated value we seek.

%%%%%%%%%%%%%%%%%%

\subsubsection{Pseudocode}

%%%%%%%%%%%%%%%%%%

\begin{enumerate}
\item function \textbf{iterate}($\delta$, $d$, $\xb$, $(j_2, \dots, j_d)$, $p^{[d]:\langle 1:d \rangle}$): \\
// With argument $\delta$, this tries to return $p^{[\delta-1]: \langle d-\delta+2:d \rangle} = $ left side of~(\ref{eq:rec})
\begin{enumerate}
\item if $\delta$ == $d$: // Arrived at~(\ref{eq:iterate1}) which we can do with the points given
\begin{enumerate}
\item return $f \left ( x_1, p^{[d]:\langle 0,2:d \rangle}, p^{[d]:\langle 1,2:d \rangle}, p^{[d]:\langle 2,2:d \rangle}, p^{[d]:\langle 3,2:d \rangle} \right )$ // $= p^{[d-1]}$
\end{enumerate}
\item else: // Go a level higher using~(\ref{eq:rec})
\begin{enumerate}
\item $p^{[\delta]: \langle 0, d-\delta+2:d \rangle} = $ \textbf{iterate}($\delta+1$, $d$, $\xb$, $( j_2, \dots, j_{d-\delta+1}=0, \dots, j_d )$, $p^{[d]:\langle 1:d \rangle}$)
\item $p^{[\delta]: \langle 1, d-\delta+2:d \rangle} = $ \textbf{iterate}($\delta+1$, $d$, $\xb$, $( j_2, \dots, j_{d-\delta+1}=1, \dots, j_d )$, $p^{[d]:\langle 1:d \rangle}$)
\item $p^{[\delta]: \langle 2, d-\delta+2:d \rangle} = $ \textbf{iterate}($\delta+1$, $d$, $\xb$, $( j_2, \dots, j_{d-\delta+1}=2, \dots, j_d )$, $p^{[d]:\langle 1:d \rangle}$)
\item $p^{[\delta]: \langle 3, d-\delta+2:d \rangle} = $ \textbf{iterate}($\delta+1$, $d$, $\xb$, $( j_2, \dots, j_{d-\delta+1}=3, \dots, j_d )$, $p^{[d]:\langle 1:d \rangle}$)
\item return $f \left ( 
x_{d-\delta+1} ; 
p^{[\delta]: \langle 0,d-\delta+2:d \rangle},
p^{[\delta]: \langle 1,d-\delta+2:d \rangle},
p^{[\delta]: \langle 2,d-\delta+2:d \rangle},
p^{[\delta]: \langle 3,d-\delta+2:d \rangle}
\right )$ // $= p^{[\delta-1]}$
\end{enumerate}
\end{enumerate}

\item To start: \textbf{iterate}(1,$d$,$\xb$,$(j_2,\dots,j_d)$,$p^{[d]:\langle 1:d \rangle}$)
\end{enumerate}

%%%%%%%%%%%%%%%%%%
%%%%%%%%%%%%%%%%%%

\subsection{The derivative with respect to a point value}

%%%%%%%%%%%%%%%%%%
%%%%%%%%%%%%%%%%%%

What is the derivative with respect to a point value? 

Let the point to differentiate with respect to be:
%---------------
\begin{equation}
p^{[d]: \langle k_1, \dots, k_d \rangle}
\end{equation}
%---------------
Then we seek:
%---------------
\begin{equation}
\frac{\partial p^{[0]}}{\partial p^{[d]: \langle k_1, \dots, k_d \rangle}}
\end{equation}
%---------------

Using~(\ref{eq:soln}) and the chain rule:
%---------------
\begin{equation}
\frac{\partial p^{[0]}}{\partial p^{[d]: \langle k_1, \dots, k_d \rangle}}
=
\sum_{j_d=0}^3 \frac{\partial f \left ( 
x_d ; 
p^{[1]: \langle 0 \rangle},
p^{[1]: \langle 1 \rangle},
p^{[1]: \langle 2 \rangle},
p^{[1]: \langle 3 \rangle}
\right )
}{
\partial p^{[1]: \langle j_d \rangle}
}
\frac{
\partial p^{[1]: \langle j_d \rangle}
}{
\partial p^{[d]: \langle k_1, \dots, k_d \rangle}
}
\end{equation}
%---------------
The first term can be evaluated using~(\ref{eq:derivValue}). We immediately notice an important property: the interpolation is linear in the point values, such that the first term does not depend on them. This greatly reduces the complexity - we use the notation from~(\ref{eq:derivValue}):
%---------------
\begin{equation}
\frac{ 
\partial f(x_d)
}{
\partial p_{j_d}
}
\end{equation}
%---------------
to denote the derivative, giving:
%---------------
\begin{equation}
\frac{\partial p^{[0]}}{\partial p^{[d]: \langle k_1, \dots, k_d \rangle}}
=
\sum_{j_d=0}^3 
\frac{ 
\partial f(x_d)
}{
\partial p_{j_d}
}
\frac{
\partial p^{[1]: \langle j_d \rangle}
}{
\partial p^{[d]: \langle k_1, \dots, k_d \rangle}
}
\end{equation}
%---------------

The numerator of the second term is one dimension higher than the left hand side; this then is another recursion relation. Going another level gives:
%---------------
\begin{equation}
\frac{\partial p^{[0]}}{\partial p^{[d]: \langle k_1, \dots, k_d \rangle}}
=
\sum_{j_{d-1}=0}^3
\sum_{j_d=0}^3
% Level 1 
\frac{ 
\partial f(x_d)
}{
\partial p_{j_d}
}
% Level 2
\frac{ 
\partial f(x_{d-1})
}{
\partial p_{j_{d-1}}
}
% Remainder
\frac{
\partial p^{[2]: \langle j_{d-1}, j_d \rangle}
}{
\partial p^{[d]: \langle k_1, \dots, k_d \rangle}
}
\end{equation}
%---------------
After $d$ such recursions:
%---------------
\begin{equation}
\begin{split}
\frac{\partial p^{[0]}}{\partial p^{[d]: \langle k_1, \dots, k_d \rangle}}
&=
\sum_{j_1,\dots,j_d} 
\left (
\prod_{\alpha=1}^{d} 
\frac{ 
\partial f(x_{d-\alpha+1})
}{
\partial p_{j_{d-\alpha+1}}
}
\right ) 
\left (
\frac{
\partial p^{[d]: \langle j_1,\dots,j_d \rangle}
}{
\partial p^{[d]: \langle k_1, \dots, k_d \rangle}
}
\right ) \\
&=
\sum_{j_1,\dots,j_d} 
\left (
\prod_{\alpha=1}^{d} 
\frac{ 
\partial f(x_{d-\alpha+1})
}{
\partial p_{j_{d-\alpha+1}}
}
\right ) 
\left (
\prod_{\beta=1}^d \delta_{j_\beta, k_\beta}
\right ) \\
&=
\prod_{\alpha=1}^{d} 
\frac{ 
\partial f(x_{d-\alpha+1})
}{
\partial p_{k_{d-\alpha+1}}
}
\end{split}
\end{equation}
%---------------
This is in fact quite easy to evaluate, and does not require recursion!

%%%%%%%%%%%%%%%%%%
%%%%%%%%%%%%%%%%%%

\subsection{The derivative with respect to $x$}

%%%%%%%%%%%%%%%%%%
%%%%%%%%%%%%%%%%%%

We seek the derivative with respect to the $k$-th component $x_k$ of $\xb$:
%---------------
\begin{equation}
\frac{\partial p^{[0]}}{\partial x_k}
\end{equation}
%---------------
Using~(\ref{eq:soln}):
%---------------
\begin{equation}
\frac{\partial p^{[0]}}{\partial x_k}
=
\frac{\partial f \left ( 
x_d ; 
p^{[1]: \langle 0 \rangle},
p^{[1]: \langle 1 \rangle},
p^{[1]: \langle 2 \rangle},
p^{[1]: \langle 3 \rangle}
\right )
}{
\partial x_k
}
\end{equation}
%---------------

If $k=d$, the problem is trivially the 1D case given by~(\ref{eq:derivAbs}).

If $k<d$, then using the chain rule:
%---------------
\begin{equation}
\frac{\partial p^{[0]}}{\partial x_k}
=
\sum_{j_d=0}^3 \frac{\partial f \left ( 
x_d ; 
p^{[1]: \langle 0 \rangle},
p^{[1]: \langle 1 \rangle},
p^{[1]: \langle 2 \rangle},
p^{[1]: \langle 3 \rangle}
\right )
}{
\partial p^{[1]: \langle j_d \rangle}
}
\frac{
\partial p^{[1]: \langle j_d \rangle}
}{
\partial x_k
}
\end{equation}
%---------------
We again notice as before that the first term can be evaluated using~(\ref{eq:derivValue}) and does not depend on the point values $p$. Using the notation from~(\ref{eq:derivValue}):
%---------------
\begin{equation}
\frac{ 
\partial f(x_d)
}{
\partial p_{j_d}
}
\end{equation}
%---------------
gives:
%---------------
\begin{equation}
\frac{\partial p^{[0]}}{\partial x_k}
=
\sum_{j_d=0}^3 
\frac{\partial f(x_d)}{\partial p_{j_d}}
\frac{
\partial p^{[1]: \langle j_d \rangle}
}{
\partial x_k
}
\end{equation}
%---------------

This again defines a recursion relation - more generally, differentiating~(\ref{eq:rec}) gives:
%---------------
\begin{equation}
\frac{\partial p^{[\delta-1]: \langle d-\delta+2:d \rangle}}{\partial x_k} 
= 
\sum_{j_{d-\delta+1} = 0}^3
\frac{\partial f(x_{d-\delta+1})}{\partial p_{j_{d-\delta+1}}}
\frac{\partial p^{[\delta]: \langle d-\delta+1:d \rangle}}{\partial x_k}
\label{eq:recDer}
\end{equation}
%---------------
after $d-k$ such recursions:
%---------------
\begin{equation}
\frac{\partial p^{[0]}}{\partial x_k}
=
\sum_{j_{k+1},\dots,j_d}
\left (
\prod_{\alpha=1}^{d-k}
\frac{\partial f(x_{d-\alpha+1})}{\partial p_{j_{d-\alpha+1}}}
\right )
\frac{
\partial p^{[d-k]: \langle j_{k+1} : j_d \rangle}
}{
\partial x_k
}
\label{eq:solnDeriv}
\end{equation}
%---------------
which can be evaluated by noting that:
%---------------
\begin{equation}
\frac{
\partial p^{[d-k]: \langle j_{k+1} : j_d \rangle}
}{
\partial x_k
}
=
\frac{
\partial f \left ( 
x_k ; 
p^{[d-k+1]: \langle 0, j_{k+1}:j_d \rangle},
p^{[d-k+1]: \langle 1, j_{k+1}:j_d \rangle},
p^{[d-k+1]: \langle 2, j_{k+1}:j_d \rangle},
p^{[d-k+1]: \langle 3, j_{k+1}:j_d \rangle}
\right )
}{
\partial x_k
}
\label{eq:recDerFinal}
\end{equation}
%---------------
This can be evaluated using the 1D result~(\ref{eq:derivAbs}) for the second term; unfortunately, this requires $k-1$ levels of recursion to determine the $p^{[d-k+1]}$.

Also: do not forget that:
%---------------
\begin{equation}
\frac{\partial p^{[0]}}{\partial \tilde{x}_k} 
= \frac{\partial p^{[0]}}{\partial x_k} \frac{\partial x_k}{\partial \tilde{x}_k} 
= \frac{\partial p^{[0]}}{\partial x_k} \left ( s_{i_k+1}^k - s_{i_k}^k \right )^{-1}
\end{equation}
%---------------
since $\xb$ refers to a fraction between $0,1$.

%%%%%%%%%%%%%%%%%%

\subsubsection{Pseudocode}

%%%%%%%%%%%%%%%%%%

\begin{enumerate}
\item function \textbf{iterate\_deriv}($\delta$,$k$,$d$,$\xb$,$(j_{k+1}, \dots, j_d)$, $p^{[d]:\langle 1:d \rangle}$): \\
// With arg $\delta$, this evaluates $\partial p^{[\delta-1]: \langle d-\delta+2:d \rangle} / \partial x_k = $ the left hand of~(\ref{eq:recDer})
\begin{enumerate}
\item if $\delta == d-k+1$: // Evaluate the derivative on the right of~(\ref{eq:recDerFinal})
\begin{enumerate}
\item $p^{[d-k+1]: \langle 0, j_{k+1}:j_d \rangle}$ = \textbf{iterate}($d-k+2$, $d$, $\xb$, $(j_k=0, j_{k+1}, \dots, j_d)$, $p^{[d]:\langle 1:d \rangle}$) \\
// Recall that \textbf{iterate}($\delta,\dots$) returns $p^{[\delta-1]}$
\item $p^{[d-k+1]: \langle 1, j_{k+1}:j_d \rangle}$ = \textbf{iterate}($d-k+2$, $d$, $\xb$, $(j_k=1, j_{k+1}, \dots, j_d)$, $p^{[d]:\langle 1:d \rangle}$)
\item $p^{[d-k+1]: \langle 2, j_{k+1}:j_d \rangle}$ = \textbf{iterate}($d-k+2$, $d$, $\xb$, $(j_k=2, j_{k+1}, \dots, j_d)$, $p^{[d]:\langle 1:d \rangle}$)
\item $p^{[d-k+1]: \langle 3, j_{k+1}:j_d \rangle}$ = \textbf{iterate}($d-k+2$, $d$, $\xb$, $(j_k=3, j_{k+1}, \dots, j_d)$, $p^{[d]:\langle 1:d \rangle}$)
\item return $\partial f \left ( 
x_k ; 
p^{[d-k+1]: \langle 0, j_{k+1}:j_d \rangle},
p^{[d-k+1]: \langle 1, j_{k+1}:j_d \rangle},
p^{[d-k+1]: \langle 2, j_{k+1}:j_d \rangle},
p^{[d-k+1]: \langle 3, j_{k+1}:j_d \rangle}
\right )
/
\partial x_k
$
\end{enumerate}
\item else: // Recurse using~(\ref{eq:recDer})
\begin{enumerate}
\item $\partial p^{[\delta]:\langle 0, d-\delta+2:d \rangle} / \partial x_k$ = \textbf{iterate\_deriv}($\delta+1$,$k$,$d$,$\xb$,$(j_{k+1}, \dots, j_{d-\delta+1} = 0,\dots, j_d)$,$p^{[d]:\langle 1:d \rangle}$)
\item $\partial p^{[\delta]:\langle 1, d-\delta+2:d \rangle} / \partial x_k$ = \textbf{iterate\_deriv}($\delta+1$,$k$,$d$,$\xb$,$(j_{k+1}, \dots, j_{d-\delta+1} = 1,\dots, j_d)$,$p^{[d]:\langle 1:d \rangle}$)
\item $\partial p^{[\delta]:\langle 2, d-\delta+2:d \rangle} / \partial x_k$ = \textbf{iterate\_deriv}($\delta+1$,$k$,$d$,$\xb$,$(j_{k+1}, \dots, j_{d-\delta+1} = 2,\dots, j_d)$,$p^{[d]:\langle 1:d \rangle}$)
\item $\partial p^{[\delta]:\langle 3, d-\delta+2:d \rangle} / \partial x_k$ = \textbf{iterate\_deriv}($\delta+1$,$k$,$d$,$\xb$,$(j_{k+1}, \dots, j_{d-\delta+1} = 3,\dots, j_d)$,$p^{[d]:\langle 1:d \rangle}$)
\item return $\sum_{j_{d-\delta+1} = 0}^3
\frac{\partial f(x_{d-\delta+1})}{\partial p_{j_{d-\delta+1}}}
\frac{\partial p^{[\delta]: \langle d-\delta+1:d \rangle}}{\partial x_k}$ // = right side of~(\ref{eq:recDer})
\end{enumerate}
\end{enumerate}

\item To start: \textbf{iterate\_deriv}($1$,$k$,$d$,$\xb$,$(j_{k+1}, \dots, j_d)$,$p^{[d]:\langle 1:d \rangle}$)
\end{enumerate}

\end{document}


